\documentclass[11pt]{exam}
\usepackage{amsmath,amssymb,amsthm}
\usepackage{epsfig}
\usepackage{array}
\addtolength{\evensidemargin}{-.5in}
\addtolength{\oddsidemargin}{-.5in}
\addtolength{\textwidth}{0.8in}
\addtolength{\textheight}{0.8in}
\addtolength{\topmargin}{-.1in}
\newtheoremstyle{quest}{\topsep}{\topsep}{}{}{\bfseries}{}{ }{\thmname{#1}\thmnote{ #3}.}
\theoremstyle{quest}
\newtheorem*{definition}{Definition}
\newtheorem*{theorem}{Theorem}
\newtheorem*{question}{Question}
\newtheorem*{exercise}{Exercise}
\newtheorem*{challengeproblem}{Challenge Problem}
\newcommand{\hw}{%%%%%%%%%%%%%%%%%%%%
%% and which homework assignment is it? %%%%%%%%%
%% put the correct number below              %%%%%%%%%
%%%%%%%%%%%%%%%%%%%%%%%%%%%%%%
3
}
%%%%%%%%%%%%%%%%%%%%%%%%%%%%%%
%%%%%%%%%%%%%%%%%%%%%%%%%%%%%%
%%%%%%%%%%%%%%%%%%%%%%%%%%%%%%
\title{\vspace{-50pt}
 \Huge \name
\\
\vspace{10pt}
\huge Math 152 \thanks{Supported by NSF grant number 1317651 }\hfill Enrichment Session  \hw}
\author{}
\date{}
\pagestyle{myheadings}
% \markright{\name\hfill Homework \hw\qquad\hfill}
\newcommand{\name}{%%%%%%%%%%%%%%%%%%
 %%%%%%%%%%%%%%%%%%%%%%%%%%%%%%
 %%%%%%%%%%%%%%%%%%%%%%%%%%%%%%
 %% put your name here, so we know who to give credit to %%
Name:    -------------------------------------------- }%%%%%%%%%%%%%%%%%%%%%%%%%%%%%%
%\Huge \name
\newcommand{\Q}{\mathbb{Q}}
\newcommand{\R}{\mathbb{R}}
\newcommand{\Z}{\mathbb{Z}}
\newcommand{\C}{\mathbb{C}}
\begin{document}
\maketitle
\begin{questions}
\bracketedpoints
\question[10=4+3+3] 
Let $R$ be the region in the $xy-$ plane lying below the curve $y=\sqrt{x}\cos x$ and above that portion of the $x-$ axis with
$0\leq x\leq \frac{\pi}{2}$.
\\\\
(a) Set up, but do not yet evaluate, an integral in terms of a single variable that represents the volume
of the solid obtained by rotating $R$ about the $x$-axis. Justify your answer by drawing a picture,
labeling a sample slice, and citing the method used.
\\\\\\\\\\\\\\\\\\\\\\\\
(b) Verify that the antiderivative of the function $f(x)=x\cos 2x$ is given by $F(x)=\frac{1}{2}x\sin 2x+\frac{1}{4}\cos 2x+c$
\\\\\\\\\\\\\\\\\\\\\\
(c)Evaluate the integral of part (a).
\newpage
\question[10]
A napkin ring is made by taking a solid wooden ball (sphere) of radius $R$ and drilling a
hole of radius $a$ straight through the center. (The hole is cylindrical in shape with radius $a$, and the
resulting solid has flat edges at its top and bottom.) Find the volume of the napkin ring, in terms of
$R$ and $a$.
\newpage
\question[5]Set up, but do not evaluate an integral,  in terms of a single variable, representing the volume of the solid obtained by rotating the region enclosed by the curves 
 $y=\sqrt{x}$ and $y=x^{1/3}$  in the first quadrant around the line $x=1$. Use the method of washers and draw a picture and label a sample slice.
\newpage
\question[5] Set up, but do not evaluate {\bf an} integral,  in terms of a single variable, representing the volume of the solid obtained by rotating the region enclosed by the curves 
 $y=5-x^2$ and $y=x^2+3x+3$  about the $x$-axis. Draw a picture and label a sample slice.
\newpage
\question[5] Set up, but do not evaluate an integral,  in terms of a single variable, representing the volume of the solid obtained by rotating the region enclosed by the curves 
 $y=\sqrt{x}$ and $y=x^3$  in the first quadrant around the line $x=1$. Use the method of washers and draw a picture and label a sample slice.
\end{questions}
\end{document}