\documentclass[11pt]{exam}
\usepackage{amsmath,amssymb,amsthm}
\usepackage{epsfig}
\usepackage{array}
\addtolength{\evensidemargin}{-.5in}
\addtolength{\oddsidemargin}{-.5in}
\addtolength{\textwidth}{0.8in}
\addtolength{\textheight}{0.8in}
\addtolength{\topmargin}{-.1in}
\newtheoremstyle{quest}{\topsep}{\topsep}{}{}{\bfseries}{}{ }{\thmname{#1}\thmnote{ #3}.}
\theoremstyle{quest}
\newtheorem*{definition}{Definition}
\newtheorem*{theorem}{Theorem}
\newtheorem*{question}{Question}
\newtheorem*{exercise}{Exercise}
\newtheorem*{challengeproblem}{Challenge Problem}
\newcommand{\hw}{%%%%%%%%%%%%%%%%%%%%
%% and which homework assignment is it? %%%%%%%%%
%% put the correct number below              %%%%%%%%%
%%%%%%%%%%%%%%%%%%%%%%%%%%%%%%
2
}
%%%%%%%%%%%%%%%%%%%%%%%%%%%%%%
%%%%%%%%%%%%%%%%%%%%%%%%%%%%%%
%%%%%%%%%%%%%%%%%%%%%%%%%%%%%%
\title{\vspace{-50pt}
 \Huge \name
\\
\vspace{10pt}
\huge Math 152 \thanks{Supported by NSF grant number 1317651 }\hfill Enrichment Session  \hw}
\author{}
\date{}
\pagestyle{myheadings}
% \markright{\name\hfill Homework \hw\qquad\hfill}
\newcommand{\name}{%%%%%%%%%%%%%%%%%%
 %%%%%%%%%%%%%%%%%%%%%%%%%%%%%%
 %%%%%%%%%%%%%%%%%%%%%%%%%%%%%%
 %% put your name here, so we know who to give credit to %%
Name:    -------------------------------------------- }%%%%%%%%%%%%%%%%%%%%%%%%%%%%%%
%\Huge \name
\newcommand{\Q}{\mathbb{Q}}
\newcommand{\R}{\mathbb{R}}
\newcommand{\Z}{\mathbb{Z}}
\newcommand{\C}{\mathbb{C}}
\begin{document}
\maketitle
\begin{questions}
\bracketedpoints
\question[15] Evaluate. Show all reasoning.\\\\
(a)$\displaystyle{\int_{0}^{1} \frac{dx}{(1+\sqrt{x})^5}}$ \\\\\\\\\\\\\\\\\\\\\\
(b)$\displaystyle{\int_{3}^{8} x\sqrt{1+x}\, dx}$\\\\\\\\\\\\\\\\\\\\\\\\\\
(c)$\displaystyle{\int \frac{dx}{x \ln x \ln (\ln x)}}$

\newpage

\question[3]  Let R be the region in the xy-plane bounded by the curves
$$x = 1 - y^2\,\, \mbox{and} \,\,x = y^4- 1$$
Set up, but do not evaluate, an integral in terms of a single variable that represents the area of R. Justify your answer by drawing a picture.
\\\\\\\\\\\\\\\\\\\\\\\\\\\\\\\\\\\\\\\\\
\question[3]Set up two distinct integrals, each in terms of a single variable, representing the area of the region $R$ enclosed by the curves 
 $y=\sqrt{x}$ and $y=x^{1/3}$  in the first quadrant. Justify your answer by drawing a picture.
\newpage
\question[3]Set up, but do not evaluate, an integral in terms of a single variable representing the area of the region bounded by the curves $y=5-x^2$ and $y=x^2+3x+3$. As justification, draw a picture.
\\\\\\\\\\\\\\\\\\\\\\\\\\\\\\\\\\\\\\\\\\
\question[3] Set up two distinct integrals, each in terms of a single variable, representing the area of the region $R$ enclosed by the curves 
 $y=\sqrt{x}$ and $y=x^3$  in the first quadrant. Justify your answer by drawing a picture.
\end{questions}
\end{document}